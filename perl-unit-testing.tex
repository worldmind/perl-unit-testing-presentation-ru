\documentclass{beamer}
\usepackage[T2A]{fontenc}
\usepackage[utf8]{inputenc}
\usepackage[english,russian]{babel}
\usepackage{cite,enumerate,float,indentfirst}
\usepackage{listings}
\lstset{language=Perl}
\usetheme{Madrid}

\title{Unit testing for Perl}
\author{Alexey Shrub}
\institute{ashrub@yandex.ru}
\date{2011-04-25}
\begin{document}

%%титульная страница
\maketitle

%% Что такое модульное тестирование
\begin{frame}
%\begin{center}
\frametitle{Модульное тестирование}
\begin{itemize}
\item Автоматизированное.
\item Изолированное.
\end{itemize}
%\end{center}
\end{frame}

%% Зачем нужны тесты
\begin{frame}
\frametitle{Зачем нужны модульные тесты}
\begin{itemize}
\item Необходимая верификация (+ двойная запись).
\item Борьба с ростом энтропии (регрессом) при изменениях (= легкость рефакторинга).
\item Локализация ошибок (в отличии от интеграционных).
\item Раннее обнаружение ошибок (чем раньше тем дешевле исправление ошибки).
\item Раннее обнаружение неудобного интерфейса.
\item Документация.
\end{itemize}
\end{frame}

%% Почему мало кто их пишет?
\begin{frame}
\frametitle{Стандартные отмазки нежелающих писать тесты}
\begin{itemize}
\item Нет времени.
\item Код нетестируемый.
\item Неумею и боюсь, у меня и без тестов вроде/должно работать.
\end{itemize}
\end{frame}

%% Test::More
\begin{frame}[fragile]
\frametitle{use Test::More;}
Базовые функции
\begin{itemize}
\item ok
\item is
\item new\_ok
\item is\_deeply
\item ...
\end{itemize}
Диагностика
\begin{itemize}
\item diag
\item explain
\end{itemize}
Пример:
\begin{lstlisting}
is_deeply( $got, $expected, 'Result must be ...' )
    or diag explain $got;
\end{lstlisting}
\end{frame}

%% Минимальный пример теста
\begin{frame}[fragile]
\frametitle{Минимальный пример}
\begin{block}{Пример положительного функционального теста}
\lstinputlisting{t/simple-test.t}
\end{block}
\end{frame}

%% Запуск тестов
\begin{frame}[fragile]
\frametitle{Запуск тестов}
TAP - Test Anything Protocol
\begin{block}{Run test}
\begin{verbatim}
$ perl t/simple-test.t
1..1
ok 1 - world.mind@yahoo.com must be valid
\end{verbatim}
\end{block}
\begin{block}{Run tests with Test:Harness}
\begin{verbatim}
$ prove
t/simple-test.t .. ok   
t/use.t .......... ok   
All tests successful.
Files=2, Tests=2,  1 wallclock secs ( 0.02 usr  0.01 sys +  0.14 cusr  0.02 csys =  0.19 CPU)
Result: PASS
\end{verbatim}
\end{block}
Makefile - иногда удобнее т.к. бывает нужно для тестов добавлять пути в @INC
\end{frame}

%% Тестирование исключений
\begin{frame}{Тестирование исключений}
\begin{block}{Test::Exception}
\lstinputlisting{t/exception.t}
\end{block}
\end{frame}

%% Модуль взаимодействует с внешними объектами?
\begin{frame}{Что делать если модуль взаимодействует с внешним миром?}
\begin{itemize}
\item Пишет/читает базу
\item Обращается в web страницам/скриптам
\item Пишет/читает memcache
\item Вызывает SOAP/XML-RPC сервисы
\item и т.п.
\end{itemize}
\begin{center} 
\LARGE ?
\end{center}
\end{frame}

%% Mock/Stub/Fake
\begin{frame}{Mock/Stub/Fake}
Mock модули общего назначения
\begin{itemize}
\item Test::MockObject
\item Test::MockModule
\item Test::MockClass
\end{itemize}
Специализированные
\begin{itemize}
\item DBD::Mock
\item Test::Mock::LWP
\item Cache::Memcached::Mock
\item и т.п.
\end{itemize}
\end{frame}

%% Пример подмены LWP
\begin{frame}[allowframebreaks]{Пример подмены модуля LWP}
%\begin{block}{}
\lstinputlisting{t/mock-memcache.t}
%\end{block}
\end{frame}

%% Закончили с функциональным
\begin{frame}

\end{frame}

%% 
\begin{frame}
\end{frame}

%% 
\begin{frame}
\end{frame}

%% 
\begin{frame}
\end{frame}

%% 
\begin{frame}
\end{frame}

%% 
\begin{frame}
\end{frame}

%% 
\begin{frame}
\end{frame}

%% 
\begin{frame}
\end{frame}

%% 
\begin{frame}
\end{frame}

%% 
\begin{frame}
\end{frame}

\end {document}


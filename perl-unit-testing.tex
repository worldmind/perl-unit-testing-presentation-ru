\documentclass{beamer}
\usepackage[T2A]{fontenc}
\usepackage[utf8]{inputenc}
\usepackage[english,russian]{babel}
\usepackage{cite,enumerate,float,indentfirst}
\usepackage{listings}

\usetheme{Madrid}

\title{Unit testing for Perl}
\author{Alexey Shrub}
\institute{ashrub@yandex.ru}
\date{2011-04-25}
\begin{document}

%%титульная страница
\maketitle

%% Что такое модульное тестирование
\begin{frame}
%\begin{center}
\frametitle{Модульное тестирование}
\begin{itemize}
\item Автоматизированное.
\item Изолированное.
\end{itemize}
%\end{center}
\end{frame}

%% Зачем нужны тесты
\begin{frame}
\frametitle{Зачем нужны модульные тесты}
\begin{itemize}
\item Необходимая верификация (+ двойная запись).
\item Борьба с ростом энтропии (регрессом) при изменениях (= легкость рефакторинга).
\item Локализация ошибок (в отличии от интеграционных).
\item Раннее обнаружение ошибок (чем раньше тем дешевле исправление ошибки).
\item Раннее обнаружение неудобного интерфейса.
\item Документация.
\end{itemize}
\end{frame}

%% Почему мало кто их пишет?
\begin{frame}
\frametitle{Стандартные отмазки нежелающих писать тесты}
\begin{itemize}
\item Нет времени.
\item Код нетестируемый.
\item Неумею и боюсь, у меня и без тестов вроде/должно работать.
\end{itemize}
\end{frame}

%% Unit teste and Perl
\begin{frame}[fragile]
\frametitle{Модульное тестирование в Perl: тесты}
Модуль Test::More
\begin{block}{Test example}
\lstinputlisting[language=Perl]{t/simple-test.t}
\end{block}
\end{frame}

%% Запуск тестов
\begin{frame}[fragile]
\frametitle{Модульное тестирование в Perl: запуск тестов}
\begin{block}{Run test}
\begin{lstlisting}
$ perl t/simple-test.t
1..1
ok 1 - world.mind@yahoo.com must be valid
\end{lstlisting}
\end{block}
\begin{block}{Run tests with Test:Harness}
\begin{lstlisting}
$ prove
t/simple-test.t .. ok   
t/use.t .......... ok   
All tests successful.
Files=2, Tests=2,  1 wallclock secs ( 0.02 usr  0.01 sys +  0.14 cusr  0.02 csys =  0.19 CPU)
Result: PASS
\end{lstlisting}
\end{block}
\end{frame}

%% 
\begin{frame}
\end{frame}

%% 
\begin{frame}
\end{frame}

%% 
\begin{frame}
\end{frame}

%% 
\begin{frame}
\end{frame}

%% 
\begin{frame}
\end{frame}

%% 
\begin{frame}
\end{frame}

\end {document}

